\documentclass{article}

\title{The Autograde User's Manual}
\author{Dennis Castleberry}

\setlength{\parindent}{0pt}
\setlength{\parskip}{\baselineskip}

\begin{document}

\maketitle 
\pagebreak

\tableofcontents 
\pagebreak

\section{Introduction}

Autograde is an automated grading system for programming exercises.  It
currently has modules written for Java, C, C++, and MATLAB, and could
potentially have modules written for other languages.  Autograde itself is
written in Linux shell and Java. 

\subsection{Requirements}

The following is a list of software required by Autograde:

\begin{itemize}
\item GNU coreutils
\item OpenJDK
\item smartlog
\end{itemize}

In addition, the directory structure required by autograde is like so:

\begin{verbatim}
 grading/
  progN/
   csXXXXYY/
    code.zip
\end{verbatim}

The script can now be invoked from any directory.

\subsection{What Autograde Can Do}

Autograde can do the following batch options, in order:

\begin{itemize}
\item Pre-process
\item Compile
\item Execute
\item Post-process
\item Auto-evaluate
\end{itemize}

Pre-processing consists of making any substitutions or changes to the code via
a user-supplied script.  Compilation is the act of invoking the
language-dependent compiler on the code to create an output file.  Execution
runs the code, allowing the user to automatically feed in any desired input
contained in a feed file.  Post-processing is copying the codes, object files,
run outputs, and error logs into separate directories for evaluation.

Autograde also allows for automatic evaluation.


Manual evaluation is supported by a clean directory structure with consistent
ID-labelled items per-directory.  Any type of custom evaluation is easily
scriptable.  Automatic evaluation utilizes a user-supplied specification for
parsing and grading items inside the configuration directory,
\texttt{~/.autograde}.

\subsection{What Autograde Cannot Do}

Autograde cannot evaluate any form of output that is not able to be parsed. So
it cannot evaluate graphs, produced audio, or user interfaces. However,
post-processing module may be added which does this evaluation (using e.g.  the
\texttt{diff} command.

\section{The Autograde Shell Script}

The Autograde shell script does the bulk of the pre-evaluation work: the
pre-processing, compilation, execution, and bifurcation of output and error
into separate directories. It alos interfaces with the rest of the Autograde
code, in particular the Java parser-grader.

After you run the Autograde deployment script, a few files will be added
to your directory:

\begin{verbatim}
 bin:
  autograde
 lib:
  smartlog:
   lib
   smartlog.sh
   smartcheck.sh
   smartexec.sh
 src:
  ParseGrader.java
  ParseGrader.class
\end{verbatim}

Since \texttt{~/bin} is in your \texttt{\$PATH}, you should be able to call
autograde from the \texttt{grading/} directory, like so:

\begin{verbatim}
 csXXXX_yyy@classes:/classes/csXXXX/csXXXX_yyy/grading$ autograde
\end{verbatim}

\subsection{Options}

Options to \texttt{\textbf{autograde}} are enabled in \texttt{autograde.conf}
files in: (1) autograde configuration directory, (2) the assignment directory,
and (3) the module directory. Base-level configuration sets the grading dir,
root dir, and so forth. Module-level configuration sets source file extensions,
object file extensions, etc. Assignment-level configuration sets the module
name, and other user-specified variables. 

Input can be enabled by editing the execution stage to run while piping
input in.  For example if there is an input file ~/input.txt, it could
be piped through a \texttt{java} execution by overriding the \texttt{java.exec}
execution script in the assignment directory to include:

\begin{verbatim}
 java $classname | cat ~/input.txt
\end{verbatim}

If the user wishes to omit a stage, a blank override file may be placed
in the assignment-level directory.


\subsection{The Fully-Automatic Grader}

Running agrade as follows:

\begin{verbatim}
 autograde 
\end{verbatim}

would open the automatic grader for the \texttt{output/} directory on
\texttt{prog1}. The automatic grader utilizes the grading specification in the
\texttt{spec/} directory.

\section{The Autograde Parser-Grader}
\subsection{Grading Specification}

An example of the directory structure is as follow:

\begin{verbatim}
 ~/.autograde
   autograde.conf
   assignment/
     1240/
      prog1/
        autograde.conf
        matlab.exec   
   module/
     matlab/    
       matlab.pre
       matlab.comp
       matlab.exec
       matlab.post
\end{verbatim}

Thus each course has its own directory within the assignment-level configuration.


\subsection{Grading Grammars}

In order to grade by parsing, rules must be defined for parsing parts of the
grading item. These rules are contained in a file named
\texttt{\emph{item}.peg}, where \emph{item} is one of source, output, error,
etc.  Consider the following code output:

\begin{verbatim}
 Hello, world! 
\end{verbatim}

Other outputs may be acceptable, including those for which `hello' or 'world'
have or lack initial caps. An output may lack a comma, or use a period in place
of the exclamation mark. The following grammar rule captures these possible
outputs:

\begin{verbatim}
 hello = (H|h)ello(,|) (W|w)orld(!|)
\end{verbatim}

In this example, the rule name is \texttt{hello}.  The expression is given on
the right-hand side of the equals sign.  Parentheses are used to group
subexpressions.  The \texttt{|} operator is used for disjunction.  In the
example, \texttt{(H|h)} is used to cover both `Hello' and `hello' as correct
outputs. Likewise \texttt{(,|)} covers either the presence or lack of a comma.

\subsection{Grading Schemes}

A grading scheme file tells how many points to award to the subitem when 
the grammar expressions are successfully parsed. A grading scheme file
for the example above may look like:

\begin{verbatim}
 10: hello; "'Hello, world!' output was not correct"
\end{verbatim}

The number 10 refers to the number of points to award.  The word `hello'
following the colon refers to the expression name given in the grammar file.
Finally, after the semicolon, a comment is given which is printed to the screen
if either positive points are not added to the sum or negative points are.
Comments are mandatory.

\section{Bugs}

Autograde has a peculiarity with the Java Scanner class.  When a Scanner
object is closed, it closes System.in.  When System.in is closed, the automated
file input feed given via the \texttt{-i} option no longer feeds into the
program input stream. Typically this happens when more than one Scanner object
is used in a Java program.  To avoid this behavior, it is best to stipulate
that one Scanner object be created in main and passed to any methods which
require it.

\texttt{* Thank you for using \textbf{Autograde}.}

\end{document}
